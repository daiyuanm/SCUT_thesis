\chapter{全自动运行系统中地铁屏蔽门与列车门间异物检测应用总体设计}
第 4 章所述的列车底部异常检测模型仅能实现二分类(异常或无异常),但无法进行多分类。不同类型的异常对应不同的风险等级,为实现异常风险等级的划分,根据不同的风险等级对列车下达不同的指令,需要对列车底部的异常进行多分类。动车组序列图像的分辨率为2048×1400像素,最小异常(螺钉缺失)的尺度约为40×40像素,仅占整幅图像的0.05%,最大布条的尺度为384×293像素,占整幅图像的3.924%。动车组列车底部异常的主要特点是异常尺度多样化,多尺度目标检测是动车组列车底部异常检测的一个挑战,尤其是对小尺度目标的检测。本章结合深度学习技术,对基于卷积神经网络的动车组列车底部异常的分类模型进行了研究。
\section{引言}
\section{需求分析}
\section{应用设计}
\section{总体架构}
\section{功能架构}
\section{本章小结}