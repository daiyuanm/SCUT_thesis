\chapter{轻量化模型的设计与实际部署}
第 4 章所述的列车底部异常检测模型仅能实现二分类(异常或无异常),但无法进行多分类。不同类型的异常对应不同的风险等级,为实现异常风险等级的划分,根据不同的风险等级对列车下达不同的指令,需要对列车底部的异常进行多分类。动车组序列图像的分辨率为2048×1400像素,最小异常(螺钉缺失)的尺度约为40×40像素,仅占整幅图像的0.05%,最大布条的尺度为384×293像素,占整幅图像的3.924%。动车组列车底部异常的主要特点是异常尺度多样化,多尺度目标检测是动车组列车底部异常检测的一个挑战,尤其是对小尺度目标的检测。本章结合深度学习技术,对基于卷积神经网络的动车组列车底部异常的分类模型进行了研究。
\section{引言}
\section{问题分析}
\section{轻量化模型的设计}
\subsection{改进Focus模块}
\subsection{动态机制的设计}
\subsection{模型检测头的优化}
\section{实验与结果分析}
\subsection{实验数据集}
为了评估本章所提出算法GL-YOLO的性能,将在公开数据集PASCAL VOC及第三章所构建的FODD这两个数据集上同其他先进检测模型进行对比实验。
\subsubsection*{(1) PASCAL VOC}
PASCAL VOC\cite{pascalvoc07, pascalvoc12}是一个常用的分类、识别和检测视觉目标的基准数据集。它的特点是包含多样化的场景以及多种物体类别。PASCAL VOC提供了一整套从数据标注到算法评估的标准流程,它采用了一种名为xml的标注方法,数据集中的每一张图片对应一个xml格式的文件,其中包含了该图片的名称、图片尺寸、物体类型、位置、大小、是否完整和预测难度等信息。PASCAL VOC(2005-2012)竞赛的目标主要是进行图像目标检测,数据集中包含了生活中常见的20种物体,包括飞机、自行车、鸟、船、瓶子、公共汽车、小汽车、猫、椅子、奶牛、餐桌、狗、马、摩托车、人、盆栽、羊、沙发、火车和显示器。

PASCAL VOC有两个版本的数据集:VOC2007和VOC2012。VOC2007包含9963张标注过的图片,由train/val/test三部分组成,共标注出24,640个物体。而VOC2012是VOC2007数据集的升级版,一共有11530张图片。VOC2012的trainval/test包含08-11年的所有对应图片,并与VOC2007互斥,trainval中有11540张图片,共27450个物体。VOC2007和VOC2012数据集及二者的并集数据量对比如下表所示。
\begin{table}[htbp]
	\centering
	\small
	\caption{PASCAL VOC数据集详细信息}
	\setlength{\tabcolsep}{1.3mm}
	\begin{tabular}{ccccccccccc}
		\toprule[2pt]
		\multirow{2}[4]{*}{} & \multicolumn{2}{c}{训练集} & \multicolumn{2}{c}{验证集} & \multicolumn{2}{c}{训练与验证集} & \multicolumn{2}{c}{测试集} & \multicolumn{2}{c}{全部} \\
		\cmidrule{2-11}          & 图片数   & 目标数   & 图片数   & 目标数   & 图片数   & 目标数   & 图片数   & 目标数   & 图片数   & 目标数 \\
		\midrule
		VOC07 & 2501  & 6301  & 2510  & 6307  & 5011  & 12608 & 4952  & 12032 & 9963  & 24640 \\
		VOC12 & 5717  & 13609 & 5823  & 13841 & 11540 & 27450 & {\color{red} 11540} & {\color{red} 27450} & {\color{red} 23080} & {\color{red} 54900} \\
		总计    & 8218  & 19910 & 8333  & 20148 & 16551 & 40058 & {\color{red} 16492} & {\color{red} 39482} & {\color{red} 33043} & {\color{red} 79540} \\
		\bottomrule[2pt]
	\end{tabular}%
	\label{pascal voc}%
\end{table}%

表中,黑色字体所示数字是官方给定的,由于VOC2012数据集中测试集部分没有公布,因此红色字体所示数字为估计数据,按照PASCAL VOC通常的划分方法,即训练与验证集与测试集各占总数据量的一半。在后续的实验中,本章使用VOC2007的训练集+验证集和VOC2012的训练集+验证集训练,然后使用VOC2007的测试集测试,即在大多数论文中经常看到的07+12方法。
\subsubsection*{(2) FODD}

\subsection{评价指标}
在目标检测领域,均值平均精度(mean Average Precision, mAP)和每秒传输帧数(Frame Per Second, FPS)这两个评价指标是衡量目标检测算法准确性及检测速度的权威指标,被广泛应用于评价目标检测算法的性能。此外,由于本文所研究算法将面向实际应用,需要考虑到具体的硬件设备算力。因此,本文在此还将添加模型参数量(Parameters)、运算量(GFLOPs)作为额外的评价指标。下面给出这四个评价指标的具体计算方式:
\subsubsection*{(1) mAP}
\subsubsection*{(2) FPS}
\subsubsection*{(3) Parameters}
\subsubsection*{(4) GFLOPs}
\subsection{实验环境及超参数设置}
\subsection{FODD上的实验结果}
\subsection{PASCAL VOC上的实验结果}
\subsection{消融实验}
\section{模型的实际部署}
\subsection{x86-gpu 服务器GPU}
\subsection{x86-gpu 服务器CPU}
\subsection{arm-gpu手机GPU}
\subsection{arm-gpu手机CPU}
\section{本章小结}












